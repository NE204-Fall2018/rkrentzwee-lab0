To do the calibration, elements were picked which had well-defined single energy peaks. Using the spectra from those elements, each peak was fit with a Gaussian distribution. The centroids of those Gaussians, given in channel number, were fit to the known energies of the peaks using a linear regression. This calibration was done using Am-241 and Cs-137.

To validate the calibration, the calibration was applied to two other elements which each had several energy peaks, Ba-133 and Eu-152. Again, the peaks in the spectra were fit using a Gaussian distribution and the centroids taken as the peak energy value. The true peak energy was compared with the calculated peak energy by calculating the percent error.
