Although Knoll~\cite{knoll2010radiation} suggests fitting using a polynomial of the form $$E_i = \sum_{n-0}^N a_n D_i^n,$$ it is clear that the nonlinearity is quite small. For most purposes, the linear fit would be suitable. For energies below about 100 keV, a higher degree polynomial might be necessary to be accurate, as seen in the percent error in the 80.9 keV line from Ba-133 being much greater than the percent error at other energies. A higher order polynomial fitting would need more calibration peaks than the two used in this paper.  
